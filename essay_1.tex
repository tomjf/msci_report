\documentclass[12pt]{article} % Default font size is 12pt, it can be changed here

\usepackage{geometry} % Required to change the page size to A4
\geometry{a4paper} % Set the page size to be A4 as opposed to the default US Letter

\usepackage{graphicx} % Required for including pictures

\usepackage{float} % Allows putting an [H] in \begin{figure} to specify the exact location of the figure
\usepackage{wrapfig} % Allows in-line images such as the example fish picture

\usepackage{bm}

\usepackage{amsmath}


\linespread{1.2} % Line spacing

%\setlength\parindent{0pt} % Uncomment to remove all indentation from paragraphs

\graphicspath{{Pictures/}} % Specifies the directory where pictures are stored

\setlength\parindent{0pt}

\bibliographystyle{unsrt}  

\begin{document}

%----------------------------------------------------------------------------------------
%	TITLE PAGE
%----------------------------------------------------------------------------------------

\begin{titlepage}

\newcommand{\HRule}{\rule{\linewidth}{0.5mm}} % Defines a new command for the horizontal lines, change thickness here

\center % Center everything on the page


\textsc{\LARGE Perturbed Universes and Inflationary Models}\\[0.5cm] % Major heading such as course name
\textsc{\Large MSci. Thesis}\\[0.5cm] % Minor heading such as course title

\textsc{\large Imperial College London}\\[1.5cm] % Name of your university/college

\begin{center}
\large \emph{Author:}\\
Thomas \textsc{Fletcher} % Your name
\end{center}

\vspace{2cm}

\begin{minipage}{0.4\textwidth}
\begin{flushleft} \large
\emph{Supervisor:}\\
Prof. Jo\~{a}o \textsc{Magueijo} % Your name
\end{flushleft}
\end{minipage}
~
\begin{minipage}{0.4\textwidth}
\begin{flushright} \large
\emph{Assessor:} \\
Dr Carlo  \textsc{Contaldi} % Supervisor's Name
\end{flushright}
\end{minipage}\\[3cm]


\large \textit{A thesis submitted in fulfilment of the requirements\\ for the degree of MSci. Physics}\\[0.3cm] % University requirement text
\textit{in the}\\[0.4cm]
Theoretical Physics Department
\\ Imperial College London
\vfill % Fill the rest of the page with whitespace
{\large \today}\\[3cm] % Date, change the \today to a set date if you want to be precise

%\includegraphics{Logo}\\[1cm] % Include a department/university logo - this will require the graphicx package



\end{titlepage}

%----------------------------------------------------------------------------------------
%	TABLE OF CONTENTS
%----------------------------------------------------------------------------------------

\tableofcontents % Include a table of contents

\newpage % Begins the essay on a new page instead of on the same page as the table of contents 
\section{Acknowledgments}

blah blah

\newpage

\section{Abstract}
The aim of this project is to investigate inflationary models and discover new models that give near scale invariance in accordance with Cosmic Microwave Background data.

\newpage
%----------------------------------------------------------------------------------------
%	INTRODUCTION
%----------------------------------------------------------------------------------------

\section{Introduction} % Major section

\subsection{Cosmological Principle}

\subsection{Hubble Law}

\subsection{Cosmic Microwave Background (CMB)}

\subsection{Cosmological Problems}

\subsection{Inflation and why it is needed}

\subsection{Aims of the Project}

%----------------------------------------------------------------------
\section{Theory}

\subsection{Friedmann Equations}

\subsection{Geometry}

\subsection{Cosmological Constant}

\subsection{Friedmann Models}

\subsection{Conformal Time}

\subsection{Mukhanov Sasaki Equation}

\subsection{Power Spectrum}

\subsection{Spectral Index}

\begin{equation}\label{powerspec}
P(k)=A \left(\frac{k}{k_{0}}\right)^{n_{s}-1}
\end{equation}

\begin{equation}\label{logpowerspec}
n_{s}-1 = \frac{\log (P(k))}{\log(k)}
\end{equation}

\subsection{Spectral Index}
%--------------------------------------------------------------------

\section{Early Work}

\subsection{Numerical Work}
%---------------------------------------------------------------------

\section{Matching Conditions}

\subsection{Outline}

The matching condition is a simple analytical method that can be used to approximate the spectral index of an inflationary model to a good degree of accuracy so long as $a(\eta)$ is given. This can be achieved by assuming the growing modes that freeze out and exit the horizon are proportional to the scale factor and some function of $k$. The functional dependence on $k$ can then be calculated by matching the modes at horizon crossing. At horizon crossing the following equation is valid.

\begin{equation}\label{matchingcondition}
\left[ k^{2} - \frac{a''}{a} \right] = 0
\end{equation}

\begin{equation}\label{matchingequation}
\frac{1}{\sqrt{k}} \approx a(\eta)F(k)
\end{equation}

Furthermore

\begin{equation}\label{powerfrommatchingcondition}
P(k)= k^{3}\zeta^{2} = k^{3}\frac{v^{2}}{a^{2}} = k^{3}F(k)^{2}
\end{equation}

\subsection{Constant w}



\begin{equation}\label{generalisedscalefactorinw}
a\approx\eta^{\frac{2}{3w+1}}
\end{equation}

\begin{equation}\label{nsintermsofw}
n_{s}(w)=4\pm\frac{3(1-w)}{3w+1}
\end{equation}

\subsection{Jacobi Elliptic Function Solutions}

The matching condition method can be used to derive an expression for $a(\eta)$ that gives scale invariance. Equation \ref{matchingequation} can be expressed purely in terms of the scale factor and its derivatives with respect to conformal time using equation \ref{matchingcondition} to give the following.

\begin{equation}\label{a''acubedfirststep}
a(\eta) \approx \frac{1}{F\left(\sqrt{\frac{a''}{a}}\right)\left(\frac{a''}{a}\right)^{\frac{1}{4}}}
\end{equation}

From equations \ref{logpowerspec} and \ref{powerfrommatchingcondition} it can be seen that $F(k)\propto k^{-\frac{3}{2}}$ is required for exact scale invariance. If this condition is imposed in equation \ref{a''acubedfirststep} a second order differential equation can be found for the the scale factor $a(\eta)$.

\begin{equation}\label{aproptoacubed}
a''(\eta) \propto a(\eta)^{3}
\end{equation}

It is immediately clear to see that $ a \propto \frac{1}{| \eta |} $ is a solution.

%---------------------------------------------------------------------

\section{Hamiltonian Jacobi}

\subsection{Outline}

The evolution of scalar fields is described by the Klein \cite{goldstein} Gordon Equation

\begin{equation}
\ddot{\phi} + 3H\dot{\phi}+V'(\phi)=0
\end{equation}

\begin{equation}
H^{2}=\frac{8\pi}{3 m_{pl}^{2}}[\frac{1}{2}\dot{\phi}^{2}+V(\phi)]
\end{equation}

\section{Generalised Equation of State}

\subsection{$n_{s}=1$ (Intermediate Inflation)}

\begin{equation}
H''H - 2(H')^{2}=0
\end{equation}

\begin{equation}
H(\phi) = \frac{\alpha}{\beta + \phi} \approx \frac{1}{\phi}
\end{equation}

\subsection{$n_{s} \neq 1$}

\begin{equation}
\left( \frac{H''}{2}\right) - \left( \frac{H'^{2}}{H}\right) + \left( n_{s} - 1\right)\pi H = 0
\end{equation}

\subsection{Proof of Slow roll}

\subsection{Equation of State}

\begin{equation}
\rho_{\phi} = \left( \frac{8\pi \left(n_{s}-1\right)}{9 A^{2}}\right)p_{\phi}^{2} - \left(\frac{n_{s}+2}{3}\right)p_{\phi}
\end{equation}

\subsection{Potential}

\subsection{Number of e-foldings}

\section{Conclusion}

\bibliography{references}{}
\bibliographystyle{plain}

\end{document}